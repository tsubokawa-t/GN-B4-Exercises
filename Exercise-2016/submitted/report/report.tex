\documentclass[fleqn, 14pt]{sty/extarticlej}
\oddsidemargin=-1cm
\usepackage[dvipdfmx]{graphicx}
\usepackage{indentfirst}
\textwidth=18cm
\textheight=23cm
\topmargin=0cm
\headheight=1cm
\headsep=0cm
\footskip=1cm

\def\labelenumi{(\theenumi)}
\def\theenumii{\Alph{enumii}}
\def\theenumiii{(\alph{enumiii})}
\def\:{\makebox[1zw][l]{:}}
\usepackage{comment}
\usepackage{url}
\urlstyle{same}

%%%%%%%%%%%%%%%%%%%%%%%%%%%%%%%%%%%%%%%%%%%%%%%%%%%%%%%%%%%%%%%%
%% sty/ にある研究室独自のスタイルファイル
\usepackage{jtygm}  % フォントに関する余計な警告を消す
\usepackage{nutils} % insertfigure, figef, tabref マクロ

\def\figdir{./figs} % 図のディレクトリ
\def\figext{pdf}    % 図のファイルの拡張子


\begin{document}
%%%%%%%%%%%%%%%%%%%%%%%%%%%%
%% 表題
%%%%%%%%%%%%%%%%%%%%%%%%%%%%
\begin{center}
{\Large {\bf 平成28年度GNグループB4新人研修課題 報告書}}

\end{center}
\begin{flushright}
2016年04月21日\\

乃村研究室 坪川 友輝
\end{flushright}
%%%%%%%%%%%%%%%%%%%%%%%%%%%%
%% 概要
%%%%%%%%%%%%%%%%%%%%%%%%%%%%
\section{概要}
本資料は平成28年度GNグループB4新人研修課題の報告書である.
本資料では,課題内容,理解できなかった部分,作成できなかった機能,
および自主的に作成した機能について述べる.

\section{課題内容}
課題内容は,RubyによるSlackBotプログラムの作成である.
具体的には以下の2つを行う.
\begin{enumerate}
\item 任意の文字列を発言するプログラムの作成
\item SlackBotプログラムへの機能の追加
\end{enumerate}

本課題におけるRubyのバージョンは,2.1.5である.

\section{理解できなかった部分}
\begin{enumerate}
\item Net::HTTPクラスの仕組み
\item sinatraサーバへPOSTしたときのパラメータについて\\
  スクリプトからsinatraにJSONデータをPOSTすると,sinatraはparamsとして
  データを受け取る.このとき,データに入れ子構造や配列が
  存在するとそれらはparamsで正しく受け取れていなかった.
  具体的には以下のようなデータをPOSTした場合.
\begin{verbatim}
    {"aaa": {"bbb": "111"}, "ccc": ["222", "333"]}
\end{verbatim}
  paramsは以下のようになっていた.
\begin{verbatim}
   {"aaa"=>"{\"bbb\"=>\"111\"}", "ccc"=>"333"}
\end{verbatim}
  つまり,入れ子構造は文字列として,配列は最後の要素しか渡されていなかった.
  しかし,GitHubのWebhooksからJSONデータをPOSTする場合は,
  sinatraはparamsとしてデータを受け取らない.
  このため,request.bodyからデータを取得する必要がある.
  また,入れ子構造や配列も正しく取得できた.
  なぜ,スクリプトからPOSTする場合と,GitHubからPOSTする場合とで,
  パラメータの渡され方に違いがあるのか理解できなかった.
\end{enumerate}

\section{作成できなかった機能}
\begin{enumerate}
  \item GitHubのWebhooksを用いたpush以外のイベントへの対応
\end{enumerate}

\section{自主的に作成した機能}
以下の機能を自主的に作成した.
\begin{enumerate}
\item``weather''という発言に反応し,岡山市の天気情報を発言
\item GitHubへpushを行った時,pushされた内容を発言
\end{enumerate}

\end{document}
